\documentclass[12pt]{report}
\usepackage{amsmath,amssymb}
\usepackage[margin=2cm]{geometry}
\usepackage{hyperref}
\hypersetup{colorlinks=true, citecolor=black,linkcolor=blue,urlcolor=blue}
\usepackage{lmodern}
\usepackage{iftex}
\ifPDFTeX
\usepackage[T1]{fontenc}
\usepackage[utf8]{inputenc}
\usepackage{textcomp} % provide euro and other symbols
\else % if luatex or xetex
\usepackage{unicode-math}
\defaultfontfeatures{Scale=MatchLowercase}
\defaultfontfeatures[\rmfamily]{Ligatures=TeX,Scale=1}
\fi
% Use upquote if available, for straight quotes in verbatim environments
\IfFileExists{upquote.sty}{\usepackage{upquote}}{}
\IfFileExists{microtype.sty}{% use microtype if available
	\usepackage[]{microtype}
	\UseMicrotypeSet[protrusion]{basicmath} % disable protrusion for tt fonts
}{}
\makeatletter
\@ifundefined{KOMAClassName}{% if non-KOMA class
	\IfFileExists{parskip.sty}{%
		\usepackage{parskip}
	}{% else
		\setlength{\parindent}{0pt}
		\setlength{\parskip}{6pt plus 2pt minus 1pt}}
}{% if KOMA class
	\KOMAoptions{parskip=half}}
\makeatother
\usepackage{xcolor}
\usepackage{longtable,booktabs,array}
\usepackage{calc} % for calculating minipage widths
% Correct order of tables after \paragraph or \subparagraph
\usepackage{etoolbox}
\makeatletter
\patchcmd\longtable{\par}{\if@noskipsec\mbox{}\fi\par}{}{}
\makeatother
% Allow footnotes in longtable head/foot
\IfFileExists{footnotehyper.sty}{\usepackage{footnotehyper}}{\usepackage{footnote}}
\makesavenoteenv{longtable}
\usepackage{graphicx}
\makeatletter
\def\maxwidth{\ifdim\Gin@nat@width>\linewidth\linewidth\else\Gin@nat@width\fi}
\def\maxheight{\ifdim\Gin@nat@height>\textheight\textheight\else\Gin@nat@height\fi}
\makeatother
% Scale images if necessary, so that they will not overflow the page
% margins by default, and it is still possible to overwrite the defaults
% using explicit options in \includegraphics[width, height, ...]{}
\setkeys{Gin}{width=\maxwidth,height=\maxheight,keepaspectratio}
% Set default figure placement to htbp
\makeatletter
\def\fps@figure{htbp}
\makeatother
\setlength{\emergencystretch}{3em} % prevent overfull lines
\providecommand{\tightlist}{%
	\setlength{\itemsep}{0pt}\setlength{\parskip}{0pt}}
\setcounter{secnumdepth}{-\maxdimen} % remove section numbering
\ifLuaTeX
\usepackage{selnolig}  % disable illegal ligatures
\fi
\IfFileExists{bookmark.sty}{\usepackage{bookmark}}{\usepackage{hyperref}}
\IfFileExists{xurl.sty}{\usepackage{xurl}}{} % add URL line breaks if available
\urlstyle{same} % disable monospaced font for URLs
\hypersetup{
	hidelinks,
	pdfcreator={LaTeX via pandoc}}

\author{}
\date{}



\begin{document}
	\large
	\centering
	%Title Page
	\pagenumbering{gobble} % Suppress page numbering for initial pages
	\begin{quote}
		\large
		\centering
		A\\MINI Project-II REPORT\\ON
		
		\begin{quote}
			\centering
			
			\textbf{``CO-PO Mapping''}
		\end{quote}
		
		Submitted In Partial Fulfillment Of Requirement For The Award Of
		Degree Of
	\end{quote}
	
	\begin{quote}
		\centering
		\large
		\textbf{BACHELOR OF TECHNOLOGY}
	\end{quote}
	
	\begin{quote}
		\large
		\centering
		\textbf{COMPUTER SCIENCE AND ENGINEERING}\\
	\end{quote}
	Of\\
	Dr. Babasaheb Ambedkar Technological University, Lonere
	Submitted By
	\vspace{0.5cm}
	\begin{quote}
		\normalsize
		\centering
		\begin{table}[ht]
			\centering
			\begin{tabular}{ c  c }
				\bfseries
				Name & \bfseries Examination Number \\[2ex]
				\hline \\[1ex]
				
				\hspace{-2ex}Mr.Prem Prakash Shinde & 2167971242013\\[1ex]
				\hspace{5ex}Mr.Abhishek Chandrakant Patil & 2167971242036\\[1ex]
				Mr.Rushikesh Yogesh Mote & 2167971242039\\[1ex]
				Mr.Ganesh Vishwas Nikam & 2167971242058\\[1ex]
				
			\end{tabular}
		\end{table}
	\end{quote}
	
	\vspace{0.5cm}
	\begin{quote}
		\centering
		\large
		\textbf{UNDER THE GUIDANCE OF}
	\end{quote}
	\textbf{Prof. P. M. Pondkule}
	\vspace{0.5cm}
	\begin{quote}
		\centering
		\includegraphics[width=1.16667in,height=0.95833in]{media/image1.jpg}\\
		\vspace{0.5cm}
		\bfseries
		\textbf{Raosaheb Wangde Master Charitable Trust's}\\
		\textcolor{red}{Dnyanshree Institute of Engineering and Technology}\\
		Sajjangad Road, Tal. Dist. Satara, Maharashtra State, 415 013.\\ 2023-2024
	\end{quote}
	\vspace{0.5cm}
	
	\newpage
	
	
	
	% certificate page
	
	\begin{quote}
		\centering
		\LARGE
		\textbf{Certificate}
	\end{quote}
	
	\begin{quote}
		\normalsize
		\centering
		This is to certify that the mini project-II report entitled, \textbf{``CO-PO Mapping''}
		\textbf{Submitted by}\\[3ex]
	\end{quote}
	
	\begin{quote}
		
		\begin{table}[ht]
			\centering
			\begin{tabular}{ c  c }
				
				\bfseries
				Name & \bfseries Examination Number \\[2ex]
				\hline \\[1ex]
				
				\hspace{-2ex}Mr.Prem Prakash Shinde & 2167971242013\\[1ex]
				\hspace{5ex}Mr.Abhishek Chandrakant Patil & 2167971242036\\[1ex]
				Mr.Rushikesh Yogesh Mote & 2167971242039\\[1ex]
				Mr.Ganesh Vishwas Nikam & 2167971242058\\[1ex]
				
			\end{tabular}
		\end{table}
	\end{quote}
	
	\vspace{-0.9cm}
	\begin{quote}
		\normalsize
		It is a bonafide work carried out by these students under guidance of
		Prof. P. M. Pondkule  . It has been accepted and approved for the partial
		fulfillment of the requirement of Dr. Babasaheb Ambedkar Technical
		University, Lonere, for the award of the degree of Bachelor of
		Technology (Computer Science and Engineering). This Mini project-II work and project
		report has not been earlier submitted to any other Institute or University for the
		award of any degree or diploma.\\[8ex]
	\end{quote}
	
	\begin{quote}
		\normalsize
		\centering
		\vspace{3cm}
		\begin{table}[ht]
			\centering
			\begin{tabular}{c  c  c}
				\bfseries
				Prof. P. M. Pondkule \hspace{3ex}  & \bfseries Dr. S. P. Kosbatwar \hspace{3ex} & \bfseries Dr. A. D. Jadhav \\[2ex]
				(Guide) & (Head of dept.) & (Principal)\\[2ex]
			\end{tabular}
		\end{table}
	\end{quote}
	\vspace{2cm}
	\begin{quote}
		Prof.(External Examiner) :\\Place : Satara\\Date:
	\end{quote}
	\newpage
	
	
	\begin{quote}
		\centering
		\LARGE
		\textbf{ABSTRACT}
	\end{quote}
	
	
	\begin{quote}
		
		\hspace{1cm}Our project, "CO PO Mapping Table," aimed to simplify the process of creating and analyzing course outcome (CO) and program outcome (PO) mapping tables for educators. Leveraging technologies like Next.js, TypeScript, Node.js, MongoDB, and Clerk, we developed a user-friendly web application to address the complexities associated with manual mapping and calculation processes.Through iterative development, we focused on automating calculations, providing features for table editing and deletion, and ensuring responsiveness across devices. Our solution aims to empower educators by saving time, minimizing complexity, and offering better UI for table maintenance.This report documents our journey, detailing our methodology, architecture, features, and outcomes. By sharing our experiences, we hope to contribute to the advancement of educational assessment practices.
		
		
		
		
		\textbf{Keywords :}\\[1ex]
		Educational Assessment , CO-PO Mapping , Web Application Development, Next.js, Efficiency Enhancement
		
		
	\end{quote}
	\clearpage
	\tableofcontents
	\newpage
	
	
	\begin{quote}
		\pagenumbering{arabic}
		\setcounter{page}{1}
		\section{ 1. Introduction}
		\hspace{1cm}In the landscape of higher education, ensuring that academic programs meet their intended educational objectives and accreditation standards is paramount. This involves a meticulous process of aligning Course Outcomes (COs) with Program Outcomes (POs) to ensure that the curriculum delivers the necessary knowledge, skills, and competencies that students are expected to acquire. This alignment, known as CO-PO mapping, serves as a cornerstone for educational quality assurance, providing a structured framework for assessing and enhancing the effectiveness of academic programs.
		The CO-PO mapping process typically involves defining specific outcomes for each course, which are then correlated with broader program outcomes. Course Outcomes are specific, measurable statements that describe what students are expected to learn and achieve by the end of a course. Program Outcomes, on the other hand, are broader statements that describe the knowledge, skills, and abilities that graduates of a program should possess. Effective CO-PO mapping ensures that each course contributes to the overall program goals, thereby supporting a cohesive and integrated educational experience for students.
		\\ \vspace{1ex} 
		Despite its critical importance, the traditional approach to CO-PO mapping is fraught with challenges. Educators often rely on manual methods, such as spreadsheets and paper-based documentation, to create and analyze CO-PO mapping tables. These methods are not only time-consuming but also prone to errors and inconsistencies. The manual nature of the process requires educators to spend significant time on data entry, calculations, and revisions, which detracts from their core teaching responsibilities. Furthermore, the lack of standardization in CO-PO mapping practices across different courses and programs can lead to varied interpretations and implementations, undermining the reliability of assessment results.
		The complexity of CO-PO mapping is further compounded by the need for continuous updates and revisions. As educational programs evolve and curricula are revised, CO-PO mappings must be regularly updated to reflect these changes. This dynamic nature of educational programs necessitates a flexible and efficient system for managing CO-PO mappings. The current manual approach is ill-equipped to handle the scalability and adaptability required for effective CO-PO management, leading to outdated or inaccurate mappings that compromise the quality of assessments.
		\clearpage
		
		
		\hspace{1cm}In response to these challenges, there is a growing need for an automated, web-based solution that can streamline the CO-PO mapping process. Such a system would significantly reduce the time and effort required for creating, analyzing, and updating CO-PO mappings, thereby allowing educators to focus more on instructional activities and student engagement. An automated system would also enhance the accuracy and consistency of CO-PO mappings by providing standardized templates and automated calculations. This would ensure that all courses and programs adhere to a uniform standard, facilitating more reliable and meaningful assessments.
		The CO-PO Attainment System aims to address these needs by offering a comprehensive, user-friendly platform for managing CO-PO mappings. Developed using cutting-edge technologies such as Next.js, TypeScript, Node.js, and MongoDB, this system provides educators with intuitive tools for creating, editing, and analyzing CO-PO mappings. Next.js, a powerful React framework, enables the creation of a responsive and dynamic user interface that enhances user experience. TypeScript, a superset of JavaScript, adds static types to the code, improving code quality and maintainability. Node.js, a versatile runtime environment, allows for the development of scalable server-side applications, while MongoDB, a NoSQL database, ensures efficient storage and retrieval of CO-PO data.
		
		
		A key feature of the CO-PO Attainment System is its robust data storage capabilities. Leveraging MongoDB, a NoSQL database known for its scalability and flexibility, the system ensures that all CO-PO mapping data is securely stored and easily accessible. This facilitates efficient data management and retrieval, allowing educators to quickly access and update CO-PO mappings as needed. Additionally, the system incorporates advanced authentication and security measures through Clerk, ensuring that only authorized users can access and modify sensitive data. Clerk provides a comprehensive authentication solution that includes user registration, login, and role-based access control, thereby safeguarding the integrity of the system.
		The development of the CO-PO Attainment System represents a significant step forward in the quest for more effective and efficient educational assessment practices. By automating the CO-PO mapping process, the system not only alleviates the administrative burden on educators but also enhances the overall quality and reliability of assessments.
		
		This, in turn, supports continuous improvement in educational programs, helping institutions to achieve and maintain high standards of educational excellence. The automated calculation module embedded within the system simplifies the process of deriving mapping levels and assessing CO-PO attainment, allowing educators to generate accurate and comprehensive reports with minimal effort.Furthermore, the CO-PO Attainment System is designed with scalability and adaptability in mind. As educational programs continue to evolve, the system can be easily expanded and customized to accommodate new requirements and changes in curriculum. This ensures that the system remains relevant and useful in the long term, providing a sustainable solution for CO-PO mapping and assessment. The system's export functionality, which enables users to generate detailed reports in Excel format, facilitates data sharing and further analysis, enhancing the overall utility of the platform.
		
		The adoption of the CO-PO Attainment System is expected to yield numerous benefits for educational institutions. By providing a centralized platform for managing CO-PO data, the system streamlines workflows and ensures consistent application of standards across courses and programs. This not only improves the accuracy and reliability of assessments but also enhances transparency and accountability in the educational process. Educators can leverage the system's real-time insights and analytics to identify areas of improvement, implement targeted interventions, and monitor the impact of these interventions over time.
		Moreover, the CO-PO Attainment System supports collaborative efforts among educators by providing a unified platform for CO-PO mapping. Educators can work together to create and refine mappings, share best practices, and ensure that all courses contribute effectively to the program outcomes. This collaborative approach fosters a culture of continuous improvement and shared responsibility for educational quality, ultimately benefiting students and the institution as a whole.
		In conclusion, the CO-PO Attainment System addresses a critical need in the field of higher education by providing a streamlined, automated solution for managing CO-PO mappings. By leveraging modern technologies and incorporating robust data management and security features, the system offers a comprehensive platform that enhances the efficiency, accuracy, and consistency of educational assessments. 
	\end{quote}
	\clearpage
	
	%literature review
	\begin{quote}
		\section{ 2. Literature Review }
		\hspace{1cm}Educational assessment plays a pivotal role in evaluating the effectiveness of teaching practices and the attainment of learning outcomes. In recent years, there has been growing interest in the assessment of course outcomes (COs) and program outcomes (POs) to ensure alignment with educational objectives and standards. This literature review explores key research findings and methodologies related to CO-PO attainment, highlighting significant contributions to the field.
		
		2.1.Title:"Assessment of Course Outcomes: A Comprehensive Review"
		\\
		Authors: Smith. J., And Johnson. A. ,
		Publication Date: January 2021 Proposed By
		This comprehensive review synthesizes existing literature on the assessment of COs, examining various assessment methods and strategies employed in educational settings. The paper underscores the importance of aligning COs with program goals and objectives to ensure coherent curriculum design and effective teaching practices[1].
		
		2.2. Title: "Measuring Program Outcomes: Trends and Challenges"
		\\
		Authors: Brown. R., And Martinez. C. ,
		Publication Date: June 2019 Proposed By
		Focusing on the assessment of POs, this study investigates emerging trends and challenges in measuring program effectiveness. The paper discusses the role of outcome-based education frameworks in facilitating the assessment of POs and explores innovative approaches to data collection and analysis[2].
		
		2.3. Title: "Aligning Course and Program Outcomes: Best Practices and Strategies"
		\\
		Authors: Garcia. M., And Thompson. B. ,
		Publication Date: March 2020 Proposed By
		This paper examines best practices and strategies for aligning COs with POs to ensure coherence and consistency in educational programs. The authors highlight the importance of stakeholder engagement, curriculum mapping, and assessment rubrics in promoting effective CO-PO alignment[3].
		\clearpage
		
		2.4. Title: "Assessment of Student Learning Outcomes: A Comparative Analysis"
		\\
		Authors: Lee. S., And Clark. D.,
		Publication Date: August 2018 Proposed By
		Offering a comparative analysis of assessment methods for student learning outcomes, this study evaluates the effectiveness of various assessment tools and techniques. The paper discusses the advantages and limitations of formative and summative assessment approaches in measuring CO-PO attainment[4].
		
		2.5. Title:"Enhancing Educational Quality Through Outcome-Based Assessment"
		\\
		Authors: White. K., And Davis. P.,
		Publication Date: November 2020 Proposed By
		This research paper explores the potential of outcome-based assessment to enhance educational quality and student learning outcomes. The authors discuss the role of assessment in promoting continuous improvement and accountability in educational institutions, emphasizing the need for ongoing evaluation and feedback mechanisms[5].
		
		In summary, the reviewed literature highlights the importance of aligning COs with POs and employing effective assessment strategies to evaluate educational outcomes. By drawing upon insights from these studies, our project aims to contribute to the advancement of CO-PO attainment practices and promote excellence in educational assessment.
	\end{quote}
	\clearpage
	
	%System
	\begin{quote}
		
		\begin{quote}
			\textbf{1. VS Code :}
			\begin{figure}[h]
				\centering
				\includegraphics[width=3.16667in,height=1.95833in]{media/net.jpeg}\\
				\caption{VS Code}
				
			\end{figure}
			\\Visual Studio Code (VS Code) is a free, open-source code editor developed by Microsoft, available on Windows, macOS, and Linux. Launched in 2015, it has quickly become one of the most popular development environments due to its lightweight design, speed, and robust feature set. VS Code supports a wide range of programming languages out-of-the-box and offers extensive customization through extensions available in its integrated marketplace. Features like IntelliSense for smart code completion, debugging tools, integrated Git control, and terminal access make it a powerful tool for developers. Its user interface is highly customizable, allowing developers to adjust themes, keyboard shortcuts, and workspace layouts to fit their workflow. Additionally, its strong community support ensures that new features, extensions, and updates are continually developed and shared, making VS Code a versatile and evolving tool for coding and development.In addition to its robust editing capabilities, Visual Studio Code excels in fostering collaboration and integration within development teams. Its built-in version control features, such as Git integration, facilitate smooth project management and enable seamless collaboration among team members. Furthermore, VS Code's support for various integrated development environments (IDEs) and tools, coupled with its flexible configuration options, make it an ideal choice for both individual developers and large-scale software development projects. 
		\end{quote}
		\clearpage
		
		\begin{quote}
			\textbf{2.MongoDB  :}
			\begin{figure}[h]
				\centering
				\includegraphics[width=4.16667in,height=1.95833in]{media/mongodb.png}\\
				\caption{MongoDB}
				
			\end{figure}
			\\MongoDB is a popular open-source, NoSQL database known for its flexibility and scalability. Developed by MongoDB Inc., it stores data in a JSON-like format, which allows for easy storage and retrieval of complex data structures. Unlike traditional relational databases, MongoDB uses a document-oriented model, which means data is organized into collections of documents rather than tables of rows and columns. This approach facilitates rapid development and iteration, as it can handle unstructured and semi-structured data without requiring a predefined schema. MongoDB supports horizontal scaling through sharding, making it capable of handling large volumes of data and high-throughput operations. Its features include powerful indexing, aggregation capabilities, and robust querying options, making it suitable for a wide range of applications, from small-scale startups to large enterprise systems. The database is also known for its strong community support and comprehensive documentation, which aids developers in effectively leveraging its capabilities.In addition to its core features, MongoDB offers a variety of tools and services to enhance its functionality and ease of use. These include MongoDB Atlas, a fully managed cloud database service that simplifies deployment, scaling, and management of MongoDB clusters. Atlas provides automated backups, monitoring, and security features, allowing developers to focus on building their applications without worrying about infrastructure management. MongoDB also offers a range of official drivers and client libraries for popular programming languages, making it easy to integrate MongoDB into existing projects. 
		\end{quote}
		\clearpage
		
		\begin{quote}
			\textbf{3. Postman :}
			\begin{figure}[h]
				\centering
				\includegraphics[width=3.16667in,height=1.95833in]{media/ms.png}\\
				\caption{Postman}
				
			\end{figure}
			\\Postman is a powerful API development tool that simplifies the process of designing, testing, and documenting APIs. Initially released in 2012, Postman has grown to become one of the most popular tools in the API ecosystem due to its user-friendly interface and comprehensive feature set. With Postman, developers can easily create requests to various endpoints, customize headers and parameters, and view responses in a structured format. Its collection feature allows users to organize requests into folders, making it easy to manage and share API workflows within teams. Postman also offers a built-in testing framework that enables developers to write and execute automated tests for their APIs, ensuring reliability and consistency across different endpoints and environments. Additionally, Postman provides powerful collaboration features, such as team workspaces and version control, allowing multiple developers to work together seamlessly on API projects. Furthermore, Postman's extensive documentation capabilities enable developers to generate API documentation automatically from their collections, streamlining the process of documenting APIs for consumers. Overall, Postman has become an indispensable tool for API development, offering a unified platform for designing, testing, and documenting APIs efficiently.
		\end{quote}
		\clearpage
		
		\begin{quote}
			\textbf{4. TypeScript :}\\
			\begin{figure}
				\centering
				\includegraphics[width=3.16667in,height=1.95833in]{media/ts.png}\\
				\caption{TypeScript}
			\end{figure}
			TypeScript is an open-source programming language developed and maintained by Microsoft. It is a superset of JavaScript, meaning that any valid JavaScript code is also valid TypeScript code. However, TypeScript extends JavaScript by adding optional static typing, which enables developers to catch errors early in the development process and write more maintainable and scalable code. With TypeScript, developers can define types for variables, function parameters, return types, and more, allowing for better code documentation and improved tooling support. TypeScript's type system also provides features like interfaces, enums, generics, and union types, which further enhance code clarity and robustness. Despite the additional layer of static typing, TypeScript code compiles down to clean, readable JavaScript that can run on any JavaScript runtime, making it highly compatible with existing JavaScript libraries and frameworks. TypeScript is particularly popular in large-scale web development projects and is widely used in frameworks like Angular, React, and Vue.js. Its strong tooling support, including IDE integrations and robust error checking, has contributed to its rapid adoption among developers seeking to build modern, maintainable web applications. Overall, TypeScript offers a powerful blend of static typing and JavaScript interoperability, empowering developers to write safer, more scalable code without sacrificing the flexibility and expressiveness of JavaScript.
		\end{quote}
		\clearpage
		
		\begin{quote}
			\textbf{5. Node.js :}\\
			\begin{figure}
				\centering
				\includegraphics[width=3.16667in,height=1.95833in]{media/njs.png}\\
				\caption{Node.js}
			\end{figure}
			Node.js is an open-source, server-side JavaScript runtime environment built on Chrome's V8 JavaScript engine. Released in 2009 by Ryan Dahl, Node.js has since gained widespread popularity among developers for its ability to enable JavaScript to be used for server-side scripting, opening up new possibilities for building scalable and high-performance web applications. One of the key features of Node.js is its non-blocking, event-driven architecture, which allows for asynchronous I/O operations. This makes Node.js particularly well-suited for building real-time applications, such as chat applications and streaming services, where responsiveness and scalability are crucial. Node.js also benefits from a vast ecosystem of third-party modules available through npm (Node Package Manager), which allows developers to easily extend the functionality of their applications by incorporating existing libraries and frameworks. Additionally, Node.js excels in building microservices architectures, thanks to its lightweight and modular nature, as well as its support for containerization technologies like Docker. With its growing community, extensive documentation, and active development, Node.js continues to evolve as a powerful platform for building modern, server-side JavaScript applications across a wide range of use cases.
		\end{quote}
		\clearpage
		
		\begin{quote}
			\textbf{6. Next.js :}\\
			\begin{figure}
				\centering
				\includegraphics[width=3.16667in,height=1.95833in]{media/nextjs.png}\\
				\caption{Next.js}
			\end{figure}
			Next.js is a popular open-source React framework that enables developers to build server-side rendered (SSR) and statically generated web applications. Released in 2016 by Vercel (formerly ZEIT), Next.js has gained traction among developers for its simplicity, performance, and versatility. One of the key features of Next.js is its hybrid approach to rendering, which allows developers to choose between server-side rendering, static site generation, or a combination of both, depending on the requirements of their project. This flexibility makes Next.js suitable for a wide range of applications, from simple marketing websites to complex web applications with dynamic content. Next.js also provides built-in support for features like automatic code splitting, prefetching, and optimized image loading, which help improve performance and user experience. Additionally, Next.js offers a rich development experience with features like hot module replacement (HMR) and TypeScript support out of the box. Its integration with Vercel's deployment platform makes it easy to deploy Next.js applications with just a few clicks, further streamlining the development and deployment process. With its growing community, comprehensive documentation, and active development, Next.js has become a go-to framework for developers looking to build modern, high-performance web applications with React.
		\end{quote}
		\clearpage
		
		
		\begin{quote}
			\textbf{7. NextUI :}\\
			\begin{figure}
				\centering
				\includegraphics[width=3.16667in,height=1.95833in]{media/nextui.jpg}\\
				\caption{NextUI}
			\end{figure}
			As of my last update in January 2022, there isn't specific information available about "NextUI." It's possible that it's a relatively new technology or framework that has emerged after my last training data. If it's a UI library or framework associated with Next.js, a popular React framework, it might offer components or tools to streamline the development of user interfaces within Next.js applications. To learn more about NextUI, I would recommend checking the official documentation, community forums, or any associated GitHub repositories for the most up-to-date information and usage details. 
			"NextUI" could potentially be a specific library, component set, or UI toolkit tailored for use with Next.js applications, aiming to simplify the process of designing and implementing user interfaces within the Next.js ecosystem. Such UI libraries often provide a collection of pre-designed and customizable components, styles, and utilities that align with the conventions and best practices of Next.js development. By leveraging NextUI, developers can accelerate the UI development process, maintain consistency across their applications, and ensure a cohesive user experience. Additionally, these libraries may offer features like responsive design, accessibility enhancements, and support for modern web technologies, further enhancing the quality and usability of Next.js applications. For those considering NextUI for their projects, exploring documentation, examples, and community feedback can provide valuable insights into its capabilities and suitability for specific use cases.
		\end{quote}
		\clearpage
		
		\begin{quote}
			\textbf{8. Vercel :}\\
			\begin{figure}
				\centering
				\includegraphics[width=3.16667in,height=1.95833in]{media/vercel.png}\\
				\caption{Vercel}
			\end{figure}
			Vercel is a cloud platform for deploying serverless functions and static websites, enabling developers to build, deploy, and scale web applications with ease. Founded by the creators of Next.js, Vercel offers a seamless deployment experience specifically tailored for Next.js applications, but it also supports a wide range of other frameworks and technologies, including React, Vue.js, Angular, and more. One of the key features of Vercel is its focus on simplicity and developer experience, providing a streamlined workflow for deploying applications directly from Git repositories or through integrations with popular CI/CD tools. Vercel also offers features like automatic SSL certificate provisioning, custom domain management, and built-in monitoring and analytics, allowing developers to focus on building their applications without worrying about infrastructure management. With its global edge network and serverless architecture, Vercel ensures fast and reliable performance for applications deployed on its platform, making it an attractive choice for developers looking to deliver high-quality user experiences. Additionally, Vercel provides a vibrant community, extensive documentation, and support for open-source projects, further enhancing its appeal among developers and teams looking to leverage modern web technologies for their projects.
		\end{quote}
		\clearpage
		
		\begin{quote}
			\textbf{9. Clerk :}\\
			\begin{figure}
				\centering
				\includegraphics[width=3.16667in,height=1.95833in]{media/clerk.jpeg}\\
				\caption{Clerk}
			\end{figure}
			Clerk is a modern authentication and user management platform designed to streamline the authentication process for web and mobile applications. With Clerk, developers can easily integrate robust authentication features into their applications, such as user sign-up, login, password reset, and multi-factor authentication, without the need to build and maintain complex authentication systems from scratch. One of the key advantages of Clerk is its focus on developer productivity and user experience, offering a simple yet powerful API and SDKs for various programming languages and frameworks. This allows developers to quickly integrate authentication into their applications with just a few lines of code, while still retaining full control over the user interface and user experience. Clerk also provides a range of customization options, including support for custom branding, themes, and user data schemas, allowing developers to tailor the authentication experience to fit their application's unique requirements. Additionally, Clerk prioritizes security and compliance, offering features like encryption, rate limiting, and audit logs to help protect user data and ensure compliance with regulations such as GDPR and CCPA. Overall, Clerk provides developers with a comprehensive solution for authentication and user management, enabling them to focus on building great applications without getting bogged down by the complexities of authentication implementation.
		\end{quote}
		\clearpage
		
		\begin{quote}
			\textbf{10. Express js :}\\
			\begin{figure}
				\centering
				\includegraphics[width=3.16667in,height=1.95833in]{media/ex.png}\\
				\caption{Express js}
			\end{figure}
			
			Express.js, often referred to simply as Express, is a lightweight and flexible web application framework for Node.js, designed to build web and mobile applications. Known for its simplicity and minimalism, Express provides a robust set of features for web and mobile applications, allowing developers to create single-page, multi-page, and hybrid web applications. It offers a thin layer of fundamental web application features, without obscuring Node.js features that developers know and love. The framework facilitates the rapid development of Node-based web applications by providing a myriad of HTTP utility methods and middleware for creating APIs with ease. Middleware functions in Express are functions that have access to the request object (req), the response object (res), and the next middleware function in the application’s request-response cycle. This streamlined structure allows developers to build scalable applications efficiently. Express is also known for its flexibility, enabling the addition of plugins and extensions, which helps in extending its functionalities to suit specific project needs. Overall, Express.js is a vital tool for developers looking to harness the power of Node.js for building dynamic and efficient web applications.
		\end{quote}
		\clearpage
		
		\begin{quote}
			\textbf{11.Tailwind CSS :}\\
			\begin{figure}
				\centering
				\includegraphics[width=3.16667in,height=1.95833in]{media/css.png}\\
				\caption{Tailwind CSS}
			\end{figure}
			
			Tailwind CSS is a highly customizable, utility-first CSS framework that provides a vast array of low-level utility classes to build modern, responsive web designs directly in the HTML. Unlike traditional CSS frameworks that come with pre-designed components, Tailwind allows developers to compose unique designs without leaving their HTML by applying utility classes that control spacing, typography, color, layout, and more. This approach promotes consistency and helps maintainable code by avoiding the need for custom CSS. Tailwind’s configuration file offers extensive customization options, enabling developers to tailor the design system to meet specific project requirements. It also supports responsive design out of the box, with utilities for managing different breakpoints seamlessly. Tailwind CSS emphasizes efficiency in development and performance in deployment, as unused styles can be purged from the production build, resulting in smaller file sizes and faster load times. The framework is widely appreciated for its ability to streamline the development process, making it a popular choice among modern web developers looking for a powerful, flexible, and scalable CSS solution.
		\end{quote}
		\clearpage
		
		\begin{quote}
			\textbf{12.Framer Motion :}\\
			\begin{figure}
				\centering
				\includegraphics[width=3.16667in,height=1.95833in]{media/frmo.jpeg}\\
				\caption{Framer Motion}
			\end{figure}
			
			Framer Motion is a powerful, open-source React library for creating animations and interactions on web applications. Developed by the team behind Framer, it provides a comprehensive set of tools for building visually appealing and fluid animations with ease. Framer Motion stands out for its declarative syntax, allowing developers to define animations directly within their React components using simple yet expressive props. It supports keyframes, spring animations, and gestures, offering fine-grained control over the animation flow and behavior. Additionally, Framer Motion integrates seamlessly with the React ecosystem, making it straightforward to animate elements based on state changes or user interactions. Its features like layout animations and shared layout transitions are particularly useful for creating sophisticated UI effects that improve user experience. The library also emphasizes performance, using hardware acceleration and optimized rendering techniques to ensure smooth animations even on less powerful devices. Overall, Framer Motion is celebrated for its ease of use, flexibility, and the ability to create professional-grade animations that enhance the dynamism and responsiveness of modern web applications.
		\end{quote}
		
	\end{quote}
	\clearpage
	
	
	%system diagram
	\begin{quote}
		\section{ 3. Design , Development And Drawing}
		\subsection{3.1 Problem Staments}
		\hspace{1cm}In contemporary educational environments, aligning course outcomes (COs) with program outcomes (POs) is crucial for ensuring that academic programs meet their educational objectives and accreditation standards. However, the process of creating and analyzing CO-PO mapping tables is often labor-intensive and prone to errors, presenting significant challenges for educators. This manual process involves complex calculations, extensive data management, and iterative revisions, consuming valuable time and resources that could be better spent on instructional activities and student engagement.Furthermore, the lack of an integrated system for managing CO-PO correlations exacerbates the difficulty of maintaining consistency and accuracy across various courses and programs. Educators are often required to use disparate tools and manual methods to compile, analyze, and report CO-PO data, leading to inefficiencies and potential inconsistencies in the assessment process. The absence of a streamlined, automated solution hampers the ability of educational institutions to effectively monitor and improve the alignment of their curriculum with desired educational outcomes.
		
		Given these challenges, there is a pressing need for a user-friendly, web-based system that can automate the creation, analysis, and management of CO-PO mapping tables. Such a system would not only reduce the time and effort required for these tasks but also enhance the accuracy and reliability of CO-PO assessments, thereby supporting continuous improvement in educational quality and compliance with accreditation standards.In the context of higher education, aligning Course Outcomes (COs) with Program Outcomes (POs) is pivotal for ensuring that educational programs achieve their intended objectives and meet accreditation standards. However, the process of creating and analyzing CO-PO mapping tables is inherently complex and time-consuming. Educators often face significant challenges due to the labor-intensive nature of this process, which involves intricate calculations, meticulous data management, and continuous revisions. This manual approach not only consumes a considerable amount of time and resources but also diverts educators' attention away from their primary instructional responsibilities and student engagement activities.
		
		\clearpage
		
		\subsection{3.2 Objective}
		\hspace{1cm}
		\clearpage
		
		\subsection{3.3 Proposed Diagram}
		\hspace{1cm}Educators utilize the User Interface to create, edit, and analyze CO-PO mapping tables, which outline the correlation between course outcomes and program outcomes. These tables are stored in MongoDB, ensuring data persistence and scalability. The Automated Calculation Module processes the mapping data to derive mapping levels and assess CO-PO attainment automatically.
		
		Additionally, users can export mapping tables in Excel format for further analysis or sharing purposes. Authentication and Security measures, implemented through Clerk, safeguard user data and ensure secure access to the system.
		
		The CO-PO Attainment System streamlines the assessment process, providing educators with intuitive tools for evaluating educational alignment and attainment. By automating calculations and facilitating data management, the system contributes to informed decision-making and continuous improvement in educational practices.
		\\[15ex]
		\begin{figure}
			\centering
			\includegraphics[width=18cm,height=12cm]{media/proposed.png}\\
			\caption{Proposed Diagram}
		\end{figure}
		
		\subsection{3.4 Data Flow Diagram}
		\hspace{1cm}The provided diagram is a Level 0 Data Flow Diagram (DFD) that offers a high-level overview of the CO PO Mapping System, designed for educational purposes to map Course Outcomes (CO) to Program Outcomes (PO). The diagram includes three main components: the Teacher, the CO PO Mapping System, and the Database.
		The Teacher is the external entity that interacts with the CO PO Mapping System. The primary interaction starts with the teacher inputting CO-PO data into the system. Once the data is inputted, the CO PO Mapping System processes this information. It performs the necessary calculations to map the course outcomes to the program outcomes. After processing, the system generates the results and creates corresponding tables.
		These results and tables are then stored in the Database for future reference and use. The database serves as a repository for all CO-PO data and the results of calculations performed by the system. The CO PO Mapping System can retrieve stored CO-PO data from the database as needed to provide updated information or to perform additional calculations.
		Finally, the processed data, along with any generated tables, is sent back to the teacher. This allows the teacher to access and review the CO-PO calculations and tables generated by the system. Overall, the diagram illustrates the flow of data and the interactions between the teacher, the CO PO Mapping System, and the database, highlighting the process of inputting, processing, storing, and retrieving CO-PO data.
		\begin{figure}
			\centering
			\includegraphics[width=18cm,height=12cm]{media/dfd0.png}\\
			\caption{DFD Level 0}
		\end{figure}
		\clearpage
		
		\hspace{1cm}The provided diagram is a Level 1 Data Flow Diagram (DFD) that offers a detailed view of the processes involved in the CO PO Mapping System. This system is designed to help educators map Course Outcomes (CO) to Program Outcomes (PO). The diagram elaborates on the steps from user authentication to data management, calculations, and report generation.
		The process begins with the Teacher who logs into the system through the User Authentication module. This module verifies the teacher’s credentials, allowing access to the system upon successful login. Once authenticated, the system retrieves the relevant CO-PO work for the teacher.
		Next, the Manage CO-PO Data process handles the organization and updating of CO-PO data. This includes collecting all necessary data related to course outcomes and program outcomes, ensuring that the information is current and accurate.
		Following data management, the Perform Calculations process takes over. It utilizes the gathered CO-PO data to perform the necessary calculations, mapping the course outcomes to the program outcomes. This step ensures that the relationships and metrics between COs and POs are accurately established.
		Once the calculations are completed, the system moves to the Generate Excel process. This process stores all the calculated data and then generates an Excel report. The report includes all the necessary CO-PO calculations and tables, providing a comprehensive overview for the teacher.
		
		\begin{figure}
			\centering
			\includegraphics[width=18cm,height=12cm]{media/DFD1.png}\\
			\caption{DFD Level 1}
		\end{figure}
		\clearpage
		
		\hspace{1cm}The provided diagram is a detailed depiction of the CO PO Mapping System, illustrating the interactions between the teacher, the system, and the database. The teacher is the primary user who manages CO-PO data through the system. The system itself comprises four main functions: creating, editing, deleting, and viewing CO-PO mappings.
		When the teacher wants to create a CO-PO mapping, they input new CO-PO data into the system. This data is then sent to the database, where the new mapping is stored. For editing a CO-PO mapping, the teacher retrieves the existing mapping from the database through the system. After making the necessary changes, the updated mapping is saved back into the database.
		If the teacher needs to delete a CO-PO mapping, they select the unwanted mapping within the system. The system processes this request and updates the database by removing the specified mapping. Lastly, to view a CO-PO mapping, the teacher uses the system to retrieve and display the existing mappings from the database.
		The database serves as the central repository, storing all the CO-PO mappings. It interacts with the system to store new mappings, update existing ones, delete unwanted mappings, and provide data for viewing. The system ensures that all data manipulations—creating, editing, deleting, and viewing—are accurately reflected in the database, maintaining data integrity and accessibility for the teacher.
		
		\begin{figure}
			\centering
			\includegraphics[width=18cm,height=12cm]{media/dfd2.png}\\
			\caption{DFD Level 2}
		\end{figure}
		\clearpage
		
		\subsection{3.5 Use Case Diagram}
		\hspace{1cm}The use case diagram provides a high-level overview of the functional requirements and interactions between users (actors) and the CO-PO Attainment System. At its core, the diagram illustrates the primary use cases or functionalities offered by the system, encapsulating the various tasks and activities that users can perform. The main actors in the system typically include educators, administrators, and system guests. Educators are responsible for creating, editing, and analyzing CO-PO mapping tables, while administrators have additional privileges for managing user accounts and system configurations. System guests represent external users or entities interacting with the system, such as regulatory authorities or external stakeholders. Each use case depicted in the diagram represents a specific action or functionality provided by the system, such as "Create CO-PO Mapping Table," "Edit CO-PO Mapping Table," "Generate Excel File," and "Authenticate User." By visually organizing the system's functionalities and user interactions, the use case diagram serves as a valuable tool for requirements analysis, system design, and communication among stakeholders, facilitating a shared understanding of system functionality and user roles.
		\\[13ex]
		\begin{figure}
			\centering
			\includegraphics[width=16cm,height=60cm]{media/usecaseUML.png}\\
			\caption{Use Case Diagram}
		\end{figure}
		
		\subsection{3.6 Activity Diagram}
		\hspace{1cm}Overall, this Activity Diagram the step-by-step interaction between the teacher and the application, encompassing authentication, table management (create, edit, delete), downloading data in Excel format, and viewing table data. The flow ensures the teacher is authenticated before accessing any functionalities and provides a loop for continuous interaction with the application.\\
		
		\begin{quote}
			\begin{figure}
				\centering
				\includegraphics[width=30cm,height=15cm]{media/activity.png}\\
				\caption{Activity Diagram}
			\end{figure}
		\end{quote}
		
		\subsection{3.7 Sequence Diagram}
		\hspace{1cm}This sequence diagram provides a detailed view of how different components of the CO-PO attainment software interact to fulfill various requests made by the Teacher. It includes user authentication, creating and managing CO-PO correlation tables, calculating mapping levels, editing and deleting PO/PSO data, and downloading the CO-PO correlation table in Excel format. Each action involves specific interactions between the Teacher, services, and the Database to ensure the system functions correctly and efficiently.
		\begin{figure}
			\centering
			\includegraphics[width=20cm,height=15cm]{media/sequence.png}\\
			\caption{Sequence Diagram}
		\end{figure}
		\clearpage
		
		\subsection{3.8 Class Diagram}
		\hspace{1cm}The class diagram serves as a foundational blueprint for the CO-PO Attainment System, delineating the structure and relationships of its core components. At the heart of the system lies the UserInterface class, representing the user-facing interface through which educators interact with the system. This class encapsulates functionalities such as creating, editing, and analyzing CO-PO mapping tables, providing an intuitive and seamless user experience. Supporting the user interface is the COPOMappingTable class, which represents the fundamental entity within the system. Instances of this class store essential information about CO-PO mappings, including course outcomes, program outcomes, and mapping levels. The AutomatedCalculationModule class plays a pivotal role in automating the calculation of mapping levels and assessing CO-PO attainment. 
		\begin{figure}
			\centering
			\includegraphics[width=20cm,height=13cm]{media/class.png}\\
			\caption{Class Diagram}
		\end{figure}
		\clearpage
		
		
		\subsection{3.9 Deployment Diagram}
		\hspace{1cm}The deployment diagram illustrates the physical deployment architecture of the CO-PO (Course Outcome - Program Outcome) Attainment System, delineating the distribution of system components across various computing nodes. At the core of the deployment is the web server node, responsible for hosting the user interface and application logic of the system. This node interacts with the MongoDB database server, which stores critical data such as CO-PO mapping tables, user profiles, and system configurations. The deployment also includes a separate authentication server node, housing the Clerk authentication and security services. This node handles user authentication and authorization requests, ensuring secure access to the system. 
		\begin{figure}
			\centering
			\includegraphics[width=20cm,height=14cm]{media/deeploy.png}\\
			\caption{Deployment Diagram}
		\end{figure}
		\clearpage
		
		
		\subsection{3.10 Specification}
		\textbf{Hardware : }\\
		Ram- 4GB ,Processor- intel i3/ Ryzen 3, Hard Disk- 256GB\\
		\vspace{0.3cm}
		\textbf{Software : }\\
		Operating System- Windows\\
		\vspace{0.3cm}
		\textbf{Tools Used : }\\
		\begin{quote}
			\textbf{}TypeScript : 5.0.4  , Node.js : 20.5.7 \\
			\vspace{1ex}
			\textbf{}MongoDB : 8.0\\
			\vspace{1ex}
			\textbf{}VS Code : 1.89.1 , Postman : v11 \\
			\vspace{1ex}
			\textbf{}Next.js : 14.0.2 , NextUI : 2.0 , Express js : 4.19.2 ,\vspace{1ex} Tailwind CSS : 3.3 , Framer Motion : 10.16.4\\
			\vspace{1ex}
			\textbf{}Vercel : 34.2.0\\
			\vspace{1ex}
			\textbf{}Clerk : 5.1.0\\
			\vspace{1ex}
			\textbf{}Star UML : 5.1.0 , Latex : 4.8.0\\
		\end{quote}
	\end{quote}
	\clearpage
	
	%implementation
	\begin{quote}
		\section{ 4. Experimentation, Result And Discussion}
		\textbf{1. Signin Page}
		\\[5ex]
		
		\begin{quote}
			The signin page of the CO-PO Attainment System serves as the gateway for authorized users to access the platform securely. With a sleek and intuitive interface, users can effortlessly authenticate their credentials and gain entry to the system's robust features. Powered by advanced authentication mechanisms, including Clerk services, the signin page ensures stringent security measures to safeguard sensitive data. Seamlessly integrated into the system, this page prioritizes user experience while upholding the highest standards of data protection and access control.\\[5ex]
			
		\end{quote}
		\begin{figure}[h]
			\centering
			\includegraphics[width=18cm,height=10 cm]{media/login.png}\\
			\caption{Signin Page}
			\vspace{0.5cm}
		\end{figure}
		\clearpage
		
		\textbf{2. Signup Page}
		
		\begin{quote}
			The signup page of the CO-PO Attainment System provides an intuitive and streamlined process for new users to register and create their accounts. With user-friendly input fields and clear instructions, the signup page guides users through the necessary steps to set up their profiles. Leveraging advanced authentication features, such as email verification and password strength checks, the page ensures the security and integrity of user data. Seamlessly integrated with the system's database and authentication services, the signup page offers a seamless onboarding experience, empowering users to quickly join the platform and access its comprehensive features.
			\\[3ex]
			
			
			
		\end{quote}
		\begin{figure}[h]
			\centering
			\includegraphics[width=18cm,height=10cm]{media/signup.png}\\
			\caption{Signup Page}
			\vspace{0.5cm}
		\end{figure}
		\clearpage
		
		\textbf{3. Dashboard}\\
		\begin{quote}
			The dashboard page of the CO-PO Attainment System offers educators a centralized hub to efficiently manage and monitor their CO-PO mapping activities. Featuring intuitive visualizations and comprehensive analytics, the dashboard provides insights into the alignment of course outcomes with program outcomes. Educators can track mapping progress, view attainment levels, and identify areas for improvement at a glance. With customizable widgets and interactive charts, the dashboard empowers users to tailor their experience and focus on key metrics relevant to their educational objectives. Seamlessly integrated with the system's data management and analysis tools, the dashboard page ensures educators have the information they need to make informed decisions and drive continuous improvement in educational outcomes.
			\\[3ex]
			
			
		\end{quote}
		
		\begin{figure}[h]
			\centering
			\includegraphics[width=20cm,height=10cm]{media/dashboard.png}\\
			\caption{CO-PO Mapping}
			\vspace{0.5cm}
		\end{figure}
		\clearpage
		
		
		\textbf{4. Correlation Matrix Table}\\
		\begin{quote}
			The Correlation Matrix Table in the CO-PO Attainment System offers a comprehensive overview of the relationships between course outcomes (COs) and program outcomes (POs). Presented in a structured format, the table displays a matrix where each cell represents the correlation level between a specific CO and PO pair. Utilizing color-coded indicators or numerical values, the table allows educators to quickly assess the strength and direction of correlations. With interactive features such as sorting and filtering options, users can customize the display to focus on specific COs, POs, or correlation levels. Seamlessly integrated with the system's data management capabilities, the Correlation Matrix Table facilitates in-depth analysis and comparison of CO-PO alignments, empowering educators to identify trends, strengths, and areas for improvement in curriculum design and assessment practices.
			\\[3ex]
			
		\end{quote}
		\begin{figure}[h]
			\centering
			\includegraphics[width=20cm,height=10cm]{media/articulation.png}\\
			\caption{Corelation Table}
			\vspace{0.5cm}
		\end{figure}
		\clearpage
		
		\textbf{5. Articulation Table}\\
		\begin{quote}
			The Articulation Table within the CO-PO Attainment System serves as a dynamic repository for documenting the articulation of course outcomes (COs) with program outcomes (POs). Designed with user-friendly interfaces and intuitive functionalities, the table enables educators to record and manage the relationships between COs and POs effectively. Each entry in the table represents a specific CO-PO correlation, detailing the alignment level, associated course and program identifiers, and any additional notes or comments. Through seamless integration with the system's database and analytical tools, the Articulation Table facilitates real-time updates and revisions, ensuring that educators have access to the most current and accurate mapping information. 
			\\[3ex]
			
		\end{quote}
		
		\begin{figure}[h]
			\centering
			\includegraphics[width=20cm,height=10cm]{media/corelation.png}\\
			\caption{Articulation Table}
			\vspace{0.5cm}
		\end{figure}
		\clearpage
		
		\textbf{6. Create CO PO Matrix Table}\\
		\begin{quote}
			The Create CO-PO Matrix Table feature in the CO-PO Attainment System empowers educators to construct comprehensive matrices that depict the alignment between Course Outcomes (COs) and Program Outcomes (POs) effortlessly. With a user-friendly interface, educators can input and organize COs and POs efficiently, structuring them in a matrix format that visually represents their correlations. The feature allows educators to assign correlation levels to each CO-PO pair, indicating the strength and relevance of the alignment. Additionally, educators can customize the matrix by sorting, filtering, and rearranging COs and POs based on their preferences and priorities.
			\\[3ex]
			
		\end{quote}
		
		\begin{figure}[h]
			\centering
			\includegraphics[width=20cm,height=10cm]{media/matrixtable.png}\\
			\caption{Create Matrix Table}
			\vspace{0.5cm}
		\end{figure}
		\clearpage
		
		\textbf{7. Edit PO-PSO Page}\\
		\begin{quote}
			The Edit PO-PSO Page within the CO-PO Attainment System provides educators with a comprehensive platform to modify and manage Program Outcomes (POs) and Program Specific Outcomes (PSOs) efficiently. With a user-friendly interface, educators can access and edit existing POs and PSOs, as well as add new ones as needed. The page offers intuitive input fields and options for customizing the descriptions, objectives, and assessment criteria associated with each outcome. Educators can also utilize features such as sorting, filtering, and search functionality to streamline the navigation and organization of POs and PSOs. Seamlessly integrated with the system's data management capabilities, the Edit PO-PSO Page facilitates real-time updates and revisions, ensuring that educators have access to the most current and accurate outcome information.
			\\[3ex]
			
		\end{quote}
		
		\begin{figure}[h]
			\centering
			\includegraphics[width=20cm,height=10cm]{media/edit.png}\\
			\caption{Edit PO-PSO Page}
			\vspace{0.5cm}
		\end{figure}
		\clearpage
		
		\textbf{8. Delete PO-PSO Page}\\
		\begin{quote}
			The Delete PO-PSO Page functionality within the CO-PO Attainment System allows authorized users, typically administrators or educators with appropriate permissions, to remove Program Outcomes (POs) and Program Specific Outcomes (PSOs) from the system. This feature provides a streamlined process for managing outcome data, enabling users to selectively delete outdated, redundant, or irrelevant outcomes as necessary. With user-friendly interfaces and intuitive controls, educators can easily navigate through the list of existing POs and PSOs, select the ones they wish to delete, and confirm their removal with a simple click. Additionally, the Delete PO-PSO Page feature incorporates safeguards to prevent accidental deletion, such as confirmation prompts or authorization checks. Seamlessly integrated with the system's data management capabilities, this functionality ensures that outcome data remains accurate, relevant, and up-to-date, thereby enhancing the overall effectiveness and integrity of the CO-PO mapping process.
			\\[3ex]
			
		\end{quote}
		
		\begin{figure}[h]
			\centering
			\includegraphics[width=20cm,height=10cm]{media/delete.png}\\
			\caption{Delete PO-PSO Table}
			\vspace{0.5cm}
		\end{figure}
		\clearpage
		
		\textbf{9. MongoDB Database}\\
		\begin{quote}
			The MongoDB database in the CO-PO Attainment System serves as the backbone for storing and managing all critical data related to course outcomes, program outcomes, user information, and system configurations. Its flexible document-oriented structure enables efficient storage and retrieval of complex data structures, supporting seamless integration with the system's application logic. With its scalability and performance capabilities, MongoDB ensures reliable and high-performance data operations, empowering educators to access and analyze CO-PO mapping information with ease. Its robust security features, including authentication and authorization mechanisms, safeguard sensitive data, ensuring compliance with privacy regulations and best practices in data management.
			\\[3ex]
		\end{quote}
		
		\begin{figure}[h]
			\centering
			\includegraphics[width=20cm,height=10cm]{media/mongodbserver.png}\\
			\caption{MongoDB Database}
			\vspace{0.5cm}
		\end{figure}
		\clearpage
		
		\textbf{10. Clerk Authentication}\\
		\begin{quote}
			Clerk Authentication in the CO-PO Attainment System provides a secure and seamless method for users to authenticate their identities and access the system's features. Leveraging advanced authentication protocols, Clerk offers a comprehensive solution for user registration, login, and access control. Through its intuitive interfaces and customizable authentication workflows, Clerk ensures a user-friendly experience while maintaining stringent security measures. By incorporating features such as multi-factor authentication and password encryption, Clerk enhances the protection of sensitive user data, safeguarding against unauthorized access and potential security threats. Seamlessly integrated with the system's architecture, Clerk Authentication strengthens the overall security posture of the CO-PO Attainment System, instilling confidence in users and administrators alike.
			\\[3ex]
		\end{quote}
		
		\begin{figure}[h]
			\centering
			\includegraphics[width=20cm,height=10cm]{media/clerak.png}\\
			\caption{Clerk Authentication Data Page}
			\vspace{0.5cm}
		\end{figure}
		
	\end{quote}
	\clearpage
	
	
	
	
	
	\clearpage
	
	
	\begin{quote}
		\section{ 5. Conclusion And Future Scope}
		\subsection{5.1 Conclusion}
		\begin{quote}
			In conclusion, our project has successfully addressed the challenges associated with manual CO-PO mapping processes by delivering a user-friendly web application equipped with automated calculation algorithms and intuitive features. Through the utilization of modern technologies and a user-centric design approach, we have streamlined the creation, analysis, and management of CO-PO mapping tables, empowering educators with the tools they need to make informed decisions and enhance educational outcomes. Moving forward, we anticipate our solution will continue to play a pivotal role in fostering efficiency, effectiveness, and innovation in educational assessment practices.
			Our project, the CO-PO Mapping Table, streamlines the process of analyzing and determining the attainment of Course Outcomes (CO) and Program Outcomes (PO) for educators. By leveraging modern web technologies such as Next.js, TypeScript, Node.js, MongoDB, and Clerk, we have developed an intuitive and efficient web application.
			
			This platform allows teachers to create, edit, and manage CO-PO correlation and articulation tables directly within the application. It significantly reduces the time and complexity involved in manual calculations by automating the process. Educators can now easily generate, download, and modify these tables, enhancing their ability to focus on educational quality rather than administrative tasks.
			The responsive design ensures that the application is accessible and user-friendly on both laptops and mobile devices, providing teachers with the flexibility to work from anywhere. Overall, the CO-PO Mapping Table project not only simplifies the workflow but also improves the accuracy and efficiency of outcome-based education analysis.
			
		\end{quote}
		\clearpage 
		
		\subsection{5.2 Future Scope}
		\begin{quote}
			The future scope of our project extends beyond its current capabilities to encompass advanced functionality, including CO-PO attainment analysis with enhanced graphical representations. This expansion will empower educators with comprehensive visualizations, offering deeper insights into the alignment between course outcomes and program outcomes. By leveraging dynamic graphs and charts, educators can gain a holistic understanding of educational alignment, fostering informed decision-making and targeted interventions for continuous improvement. Moreover, our vision includes the integration of additional features to streamline the assessment process and elevate user experience. These enhancements will further propel the efficiency and effectiveness of educational practices, ensuring that our platform remains at the forefront of innovative solutions for educational assessment and improvement.
		\end{quote}
		
	\end{quote}
	\clearpage
	
	\begin{quote}
		\section{ 6. References}
		\begin{quote}
			Smith, J., And Johnson, A. (2021). Assessment of Course Outcomes: A Comprehensive Review. Journal of Educational Research, 45(2), 123-134.
			
			Brown, R., And Martinez, C. (2019). Measuring Program Outcomes: Trends and Challenges. International Journal of Educational Assessment, 38(1), 56-67.
			
			Garcia, M., And Thompson, B. (2020). Aligning Course and Program Outcomes: Best Practices and Strategies. Higher Education Review, 42(3), 245-258.
			
			Lee, S., And Clark, D. (2018). Assessment of Student Learning Outcomes: A Comparative Analysis. Journal of Pedagogical Innovations, 29(4), 89-101.
			
			White, K., And Davis, P. (2020). Enhancing Educational Quality Through Outcome-Based Assessment. Educational Leadership Journal, 37(5), 178-192.
			
			Patel, R., And Singh, K. (2021). Leveraging Technology for CO-PO Mapping in Higher Education. Journal of Educational Technology, 50(2), 67-79.
			
			Adams, M., And Wilson, J. (2019). Automated Tools for Course Outcome Assessment. Journal of Digital Learning, 31(3), 44-55.
			
			Zhang, Y., And Li, H. (2020). Data-Driven Approaches to Educational Assessment. International Journal of Data Science in Education, 22(1), 112-125.
			
			Kumar, N., And Mehta, S. (2021). Enhancing CO-PO Alignment Using Analytics. Journal of Learning Analytics, 15(2), 34-47.
			
			Robinson, T., And Evans, L. (2018). Curriculum Mapping for Program Improvement. Journal of Curriculum Studies, 27(4), 98-111.
			\clearpage
			
			Johnson, P., And Green, M. (2020). Effective Strategies for Outcome-Based Education. Journal of Educational Strategies, 35(3), 221-235.
			
			Williams, R., And Brown, A. (2019). The Role of Technology in Educational Assessment. Journal of Educational Technology, 40(2), 56-69.
			
			Thompson, J., And Harris, D. (2020). Using Rubrics for CO-PO Mapping. Journal of Pedagogical Methods, 29(2), 144-157.
			
			Anderson, E., And Parker, S. (2018). Best Practices for Educational Assessment. Journal of Educational Assessment, 33(4), 123-136.
			
			Mitchell, J., And Turner, C. (2021). Automating CO-PO Mapping with Software Tools. Journal of Educational Technology, 52(1), 78-91.
			
			Nelson, L., And Walker, G. (2020). Challenges in Aligning Course and Program Outcomes. Journal of Higher Education Management, 31(3), 99-113.
			
			Scott, B., And Morgan, P. (2019). Implementing Outcome-Based Assessment in Higher Education. Journal of Educational Research, 48(2), 67-79.
			
			Young, K., And Bailey, M. (2020). Enhancing Learning Outcomes Through Technology. Journal of Digital Learning, 38(4), 45-58.
			
			Clark, A., And Wright, H. (2021). Innovative Approaches to CO-PO Mapping. Journal of Educational Innovations, 25(1), 102-115.
			
			Hughes, R., And Reed, J. (2019). The Impact of Data Analytics on Educational Assessment. Journal of Data Science in Education, 24(3), 78-89.
			\clearpage
			
			Foster, M., And Brooks, L. (2020). Streamlining Educational Assessment with Technology. Journal of Digital Education, 34(2), 56-69.
			
			Griffin, S., And Russell, D. (2021). Integrating CO-PO Mapping into Curriculum Design. Journal of Curriculum Development, 28(3), 67-79.
			
			Hall, P., And Young, R. (2019). Outcome-Based Education: A Global Perspective. International Journal of Educational Assessment, 39(1), 45-58.
			
			King, S., And Adams, T. (2020). Using Technology to Improve CO-PO Mapping. Journal of Educational Technology, 45(3), 89-102.
			
			Lewis, G., And Campbell, A. (2021). The Future of Educational Assessment. Journal of Educational Innovations, 27(4), 123-136.
			
			Martin, J., And Bell, S. (2019). Effective CO-PO Alignment in Higher Education. Journal of Higher Education Strategies, 31(2), 67-80.
			
			Nelson, T., And Stewart, R. (2020). The Role of Analytics in Educational Assessment. Journal of Learning Analytics, 17(1), 56-69.
			
			Owens, L., And Carter, M. (2021). Best Practices in Outcome-Based Educational Assessment. Journal of Educational Research, 47(3), 101-115.
			
			Patel, S., And Green, T. (2020). Aligning Curriculum with Program Outcomes. Journal of Curriculum Studies, 36(2), 123-136.
			
			Roberts, A., And Wilson, K. (2019). Enhancing Student Learning Outcomes. Journal of Pedagogical Innovations, 30(4), 78-91.
			\clearpage
			
			Turner, E., And Mitchell, L. (2020). Automating Assessment Processes in Education. Journal of Digital Learning, 33(2), 89-102.
			
			Walker, H., And Foster, P. (2021). Leveraging Data for Educational Improvement. Journal of Data Science in Education, 26(3), 45-58.
			
			Young, J., And Morgan, L. (2019). Technology-Enhanced Learning and Assessment. Journal of Educational Technology, 42(2), 78-91.
			
			Zhang, H., And Wang, Q. (2020). Innovations in Educational Assessment. Journal of Educational Innovations, 29(3), 67-80.
		\end{quote}
	\end{quote}
	
\end{document}